\documentclass[12pt]{report}

%%
% to enumerate subsubsection
%%
\addtocounter{tocdepth}{3}
\setcounter{secnumdepth}{3}
% \usepackage{tgtermes}
\usepackage[a4paper, margin=1in]{geometry}
\usepackage[T1]{fontenc}
\usepackage[utf8]{inputenc}
\usepackage{graphicx} 
\usepackage{tikz}
\usepackage{amsmath, amssymb}
% \usepackage{indentfirst}
\usepackage{cite}
\usepackage{natbib}
\usepackage[nameinlink,noabbrev]{cleveref}
\usepackage[capposition=top]{floatrow}
\usepackage{float}
\usepackage{caption}
\usepackage{xfrac}
\usepackage{braket}

%%
% to enumerate subsubsection
%%
\addtocounter{tocdepth}{3}
\setcounter{secnumdepth}{3}

% Snippet
\newcommand{\BSM}{Black--Scholes--Merton }
\newcommand{\lnorm}{log-normally }
\newcommand{\bmotion}{Brownian Motion }
\newcommand{\stvar}{stochastic variable }
\newcommand{\wienpro}{Wiener process }
\newcommand{\markpro}{Markov process }

% 
% Bunch of new commands
% 
% brownian motion
\newcommand{\dBm}{dW\left(t\right)}
\newcommand{\dpoiss}{dq\left(t\right)}
\newcommand{\DBm}{\delta{W\left(t\right)}}
\newcommand{\Bm}{W\left(t\right)}
\newcommand{\Bmsub}[1]{W_{#1}\left(t\right)}
\newcommand{\Dt}{\Delta t} 
\newcommand{\Bmdist}{\DBm \sim N \left( 0, \Dt \right)}
\newcommand{\ft}{f\left(t, \Bm \right)}
\newcommand{\E}{\mathop{\mathbb{E}}}
\newcommand{\ct}{c\left(t, x\right)}
\newcommand{\dcx}{\frac{\delta\ct}{\delta x}}
\newcommand{\dciix}{\frac{\delta^2\ct}{\delta x^2}}
\newcommand{\dct}{\frac{\delta\ct}{\delta t}}
\newcommand{\N}[1]{N\left(#1\right)}
\newcommand{\dsub}[1]{d_{#1}\left(\Dt, x\right)}
\newcommand{\call}[2]{c\left( #1, #2\right)}
% % about stock
\newcommand{\St}{S\left(t\right)}
\newcommand{\Vt}{V\left(t\right)}
\newcommand{\Si}{S\left(0\right)}
\newcommand{\dSt}{dS\left(t\right)}
\newcommand{\DSt}{\Delta S\left(t\right)}
\newcommand{\dSr}{\frac{\dSt}{\St}}
\newcommand{\DSr}{\frac{\DSt}{\St}}
\newcommand{\Scontinuous}{\St = \Si e^{\sigma\Bm + \left(\alpha - \frac{1}{2 \sigma^2}\right)t}}
\newcommand{\Itobmdiff}{d\ft = \left[\frac{\partial \ft }{\partial t} + \frac{1}{2} \frac{\partial ^2\ft }{\partial x^2}\right]dt + \frac{\partial \ft}{\partial x} \dBm}
\newcommand{\Scontinousdiff}{d\St &= \alpha \St dt + \sigma \St \dBm}
\newcommand{\Scontinuousrate}{\dSr &= \alpha dt + \sigma \dBm}
 \newcommand{\Sdiscretediff}{d\St &= \alpha \St \Dt + \sigma \St \dBm}
\newcommand{\Sshort}{\Si e^{X}}
\newcommand{\CCRdist}{X \sim N\left(\left(\alpha - \frac{1}{2}\sigma^2\right)t, \sigma^2 t\right)}
\newcommand{\Sdiscreterate}{\DSr &= \alpha \Delta t + \sigma \DBm}
\newcommand{\Sdiscreterateexp}{\E \DSr = \alpha \Dt}
\newcommand{\Sdiscreteratevar}{var \DSr = \sigma ^2 \Dt}
\newcommand{\Sdiscreteratedist}{\DSr \sim N\left(\alpha\Dt, \sigma\Dt\right)}
\newcommand{\Sexp}{\E\St = \Si e^{\alpha t}}
\newcommand{\Svar}{var\St = \Si^2 e^{2\alpha t}\left(e^{\sigma^2 t} - 1\right)}
\newcommand{\Sshortt}{\St = \Si e^{X t}}
\newcommand{\CCRt}{X = \frac{1}{t} \ln{\frac{\St}{\Si}}}
\newcommand{\CCRtdist}{X \sim N\left(\alpha - \frac{\sigma^2}{2}, \frac{\sigma^2}{t}\right)}
\newcommand{\BSMpde}{\dct + r x \dcx + \frac{1}{2} \sigma^2 x^2 \dciix = r\ct}
\newcommand{\BSMeq}[1]{r\call{t}{#1} = \frac{\partial \call{t}{#1}}{\partial t} + r #1 \frac{\partial \call{t}{#1}}{\partial #1} + \frac{1}{2} \sigma ^2 #1 ^2 \frac{\partial ^2 \call{t}{#1}}{\partial #1 ^2}}
\newcommand{\BSMGreeks}[1]{r\call{t}{#1} = \Theta + r #1 \Delta + \frac{1}{2} \sigma ^2 #1 ^2 \Gamma}
\newcommand{\BSMsol}{\ct &= x\N{\dsub{+}} - K e^{-r\Dt} \N{\dsub{-}}}
\newcommand{\dpm}{\dsub{\pm} &= \frac{1}{\sigma\sqrt{\Dt}} \left[\log\frac{x}{K} + \left(r \pm \frac{\sigma^2}{2}\Dt\right)\right]}

% CIR Stock price Stochastic process
\newcommand{\HSVstock}{
  d\St &= \alpha \St dt + \sqrt{\Vt} \St d \Bmsub{S}
}

% CIR volatility
\newcommand{\HSVvol}{
  d\Vt &= \kappa\left(\theta - \Vt \right) dt + \sigma \sqrt{\Vt} d \Bmsub{V}
}
\newcommand{\dportfolio}{dX\left(t\right) &= \Delta\left(t\right) d\St + r \left(X\left(t\right) - \Delta\left(t\right) \St \right) dt}

\usepackage{Sweave}
\begin{document}
\Sconcordance{concordance:index.tex:index.Rnw:%
1 18 1 1 0 13 1}
\Sconcordance{concordance:index.tex:./State/index.Rnw:ofs 33:%
1 55 1}
\Sconcordance{concordance:index.tex:index.Rnw:ofs 89:%
34 4 1}

\tableofcontents{}



%%%%%%%%%%%%%%%%%%%%%%%%%%%%%%%%%%%%%%%%%%%%%%%%%%%%%%%%%%%%%%%%%%%%%%%%%%%%%%%%
%
%  CHAPTER: Introduction
%
%%%%%%%%%%%%%%%%%%%%%%%%%%%%%%%%%%%%%%%%%%%%%%%%%%%%%%%%%%%%%%%%%%%%%%%%%%%%%%%%
\chapter*{Introduction}
\label{cha:Introduction}
\addcontentsline{toc}{chapter}{Introduction}

Talk about what is done to price a vanilla option throuhout the BSM method.
How does the BSM model is fair under its assumption. What about if we are going beyond ?
How perfomant is it ? 
What about other model such as \ldots ?

Using R. \cite{R}
% \SweaveInput{upstream}
% \SweaveInput{underlying}
% \SweaveInput{BSM}
% \SweaveInput{models}
%%%%%%%%%%%%%%%%%%%%%%%%%%%%%%%%%%%%%%%%%%%%%%%%%%%%%%%%%%%%%%%%%%%%%%%%%%%%%%%%
%
%  CHAPTER:Methodology
%
%%%%%%%%%%%%%%%%%%%%%%%%%%%%%%%%%%%%%%%%%%%%%%%%%%%%%%%%%%%%%%%%%%%%%%%%%%%%%%%%
\chapter{Methodology}
\label{cha:Methodology}


%%%%%%%%%%%%%%%%%%%%%%%%%%%%%%%%%%%%%%%%%%%%%%%%
% SECTION: General overview
%%%%%%%%%%%%%%%%%%%%%%%%%%%%%%%%%%%%%%%%%%%%%%%%
\section{General overview}
\label{sec:methodology:general}

My work is roughly cut into three parts, that is to say, the upstream analysis, the analysis and a discussion.

Firstly I will perform an upstream analysis meant to make the machinery (i.e. the algorithms I have created using \cite{R}) works together with market-calibrated data.
The whole step is altogether explained in \cref{sec:methodology:market}.

Afterward comes the analysis part where the delta--hedging rule will be measured according to different scenarios.
Latter will only focus on the benchmarking of hedges involving european short call derivatives priced across the Black--Scholes--Meron formula.
In order to construct such experiments, I will create as a collection of call prices, distinguished among themselves by different triples, each comprised of a maturity date ($T$), a stike price ($K$) and a volatility factor ($\sigma$).
\cref{eq:methodology:general:call} vanishes ambiguity by providing the mathematical counterpart of the previous statement.

\begin{align}
C \equiv \left\{ c_{t s k} \right\} \forall & t \in T \equiv \left\{ \frac{1}{4}, \frac{1}{2}, \frac{3}{4}, 1\right\}, \label{eq:methodology:general:call}\\
& s \in \sigma \equiv \left\{ \sigma_i \right\}_{i \in \set{5\%, 25\%, 50\%, 75\%, 95\% }}, \nonumber \\
& k \in K \equiv 
\left\{ \begin{array}{l}
\text{deep--in--the--money},\\
\text{in--the--money},\\
\text{at--the--money},\\
\text{out--of--the--money},\\
\text{deep--out--of--the--money}
\end{array} \right\} \nonumber
\end{align}
Where $(T, \sigma, k)$ are respectively the set of the considered maturities, volatilities and strike price. The subscript of the volatility set stands for the percentiles kept from the upstream analysis. [NOT GOOD]

So far, it lacks some time series, representing the evolution of the underlying over time, in order to construct the hedges.
To do so, I will implement the different techinques convered in the previous chapters, namely the geometric Brownian motion (GBM), the Merton jump diffusion (MJD) and the Heston stochastic volatility (HSV) processes and I will feed each of them with sets of arguments being relevant with the whole subset of distict double $(T, \sigma)$ constructed from the collection (\ref{eq:methodology:general:call}) so that the generated sampled series shall be pairwise consistent with one and only atoms of the set \ref{eq:methodology:general:call}.

Under the hood, the experiments on delta--hedging will be each constructed over 500 trials.
For instance, in order to measure the hedge of a european call stroke at \$ 40 with a maturity of 3 months and an estimated volatility of .20, I will store in a data table, 500 differents times series of sampled GBM computed in accordance with all members of $\Set{(t, s) | \forall t \in T, s \in \sigma}$.
Nevertheless, even though the number of 500 trials is purely arbitrary, it sounds a good trade--off between a reasonable computational time needed to run the analysis  and obtaining a big large view of the underpinned data which is broad enough to decipher trends.

The final step of the analysis will be the performance measure of each specific hedging scenario.
This quantitative assessment will be performed through a purposely tool I have developed.
More details of the latter is given at \cref{sec:methodology:hedging}.

Ultimately though, the discussion part comes where the results grouped by hypothesis will be commented.
Let's first begin with the data calibration.

%%%%%%%%%%%%%%%%%%%%%%%%%%%%%%%%%%%%%%%%%%%%%%%%
% SECTION: Market data
%%%%%%%%%%%%%%%%%%%%%%%%%%%%%%%%%%%%%%%%%%%%%%%%
\section{Market data}
\label{sec:methodology:market}

In order to correctly feed the developed models, an upstream analysis is needed to define a range of values to take as arguments.
Latter can be subdivided by their belonging model.

%%%%%%%%%%%%%%%%%%%%%%%%%%%%%%%%%%%%%%%%%%%%%%%%
% SUBSECTION: Arguments for geometric Brownian motion
%%%%%%%%%%%%%%%%%%%%%%%%%%%%%%%%%%%%%%%%%%%%%%%%
\subsection{Arguments for geometric Brownian motion}
\label{sub:methodology:market:geometric}

The drift and volatility rates, respectively $\alpha$ and $\sigma$, will be estimated using market data, especially the stocks from the S\&P500.

First and foremost, daily data will be downloaded to construct the set of vectors $S_s$, where the subscript $s$ denotes a specific symbol such as AAPL, for Apple.
Next, the algorithm will compute the daily basis log--return on each series belongong to $S_s$, as stated by \cref{eq:methodology:market:geometric:log}.

\begin{align}
\ln \frac{S_{s, i+1}}{S_{s, i}} \label{eq:methodology:market:geometric:log}
\end{align}
Because it exists some days where prices are not quoted, it is therefore necessary to be sure that the indexes $i$ and $i + 1$ are together separated by only one calendar day. This requirement provides the log-returns on the same basis and therefore make it comparable.

Once the log--returns will have been computed, \crefrange{eq:upstream:logreturn:sample}{eq:upstream:volatility:estimator} give a method to compute an estimation of the variance rate.

[TODO:+++]

\begin{align}
\hat{\sigma} \equiv \Set{\sigma_p | p \in \set{5\%, 25\%, 50\%, 75\%, 95\%}}
\end{align}
Where p are percentiles.


\begin{align}
\hat{param} \equiv \Set{\left(\alpha, \sigma\right)_p | p \in \set{5\%, 25\%, 50\%, 75\%, 95\%}}
\end{align}
By linear model.


[TODO:---]

%%%%%%%%%%%%%%%%%%%%%%%%%%%%%%%%%%%%%%%%%%%%%%%%
% SUBSECTION: Arguments for Merton Jump--diffusion process
%%%%%%%%%%%%%%%%%%%%%%%%%%%%%%%%%%%%%%%%%%%%%%%%
\subsection{Arguments for Merton Jump--diffusion process}
\label{sub:methodology:market:merton}

The arguments to be assessed for the current model come in two distinctive flavours; the ones describing the normal move in the stock prices $(\alpha, \sigma)$ and those defining the frequency $(\lambda)$ and the intensity $(\mu, \delta^2)$ of jumps.




%%%%%%%%%%%%%%%%%%%%%%%%%%%%%%%%%%%%%%%%%%%%%%%%
% SECTION: Scenarios
%%%%%%%%%%%%%%%%%%%%%%%%%%%%%%%%%%%%%%%%%%%%%%%%
\section{Scenarios}
\label{sec:methodology:scenarios}



%%%%%%%%%%%%%%%%%%%%%%%%%%%%%%%%%%%%%%%%%%%%%%%%
% SECTION: Hedging performance tool
%%%%%%%%%%%%%%%%%%%%%%%%%%%%%%%%%%%%%%%%%%%%%%%%
\section{Hedging performance tool}
\label{sec:methodology:hedging}


%%%%%%%%%%%%%%%%%%%%%%%%%%%%%%%%%%%%%%%%%%%%%%%%
% SECTION: Group construction
%%%%%%%%%%%%%%%%%%%%%%%%%%%%%%%%%%%%%%%%%%%%%%%%
\section{Group construction}
\label{sec:methodology:group}


%%%%%%%%%%%%%%%%%%%%%%%%%%%%%%%%%%%%%%%%%%%%%%%%
% SECTION: Analysis
%%%%%%%%%%%%%%%%%%%%%%%%%%%%%%%%%%%%%%%%%%%%%%%%
\section{Analysis}
\label{sec:methodology:analysis}


%%%%%%%%%%%%%%%%%%%%%%%%%%%%%%%%%%%%%%%%%%%%%%%%
% SECTION: Discussion
%%%%%%%%%%%%%%%%%%%%%%%%%%%%%%%%%%%%%%%%%%%%%%%%
\section{Discussion}
\label{sec:methodology:discussion}












\bibliography{bibl.bib}
\bibliographystyle{plainnat}
\end{document}
